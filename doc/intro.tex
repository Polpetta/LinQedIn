\section{Introduzione}

\subsection{Ambiente di sviluppo}

La realizzazione di LinQedIn \`e avvenuta nel seguente ambiente:
\begin{itemize}

\item Sistema operativo di test e sviluppo: ArchLinux x86\_64
  (rolling release)
\item Compilatore: GCC versione 5.2.0
\item Qt: versione 5.5.0
\item QMake: versione 3.0
\item Qt Creator: versione 3.4.2

\end{itemize}

 
\subsection{Descrizione del progetto}

Come da consegna, il progetto si propone di ricreare le principali
funzionalit\`a di LinkedIn. Sono presenti 3 tipi di iscrizione al servizio:
iscrizione come membro Basic, Business o Executive. L'amministrazione del
sistema \`e gestita da un Admin. Gli iscritti al servizio possono compiere
le seguenti azioni:
\begin{itemize}

\item Modifica del proprio profilo, con possibilit\`a di aggiungere
  i propri Hobby, i propri Interessi e le proprie Esperienze (lavorative)

\item Possibilit\`a di ricercare altri iscritti al servizio in base alla
  loro sottoscrizione:
  \begin{itemize}
    
  \item Basic: possibilit\`a di ricerca solamente per Nome e Cognome
  \item Business: tutte le funzionalit\`a di basic con l'aggiunta
    della ricerca per una specifica data di nascita o per un particolare
    Hobby
  \item Executive: possiede tutte le funzionalit\`a di ricerca del
    servizio. Rispetto agli iscritti Business, i membri Executive possono
    ricercare anche tramite Interessi

  \end{itemize}

\item Visualizzare i profili degli altri membri dalla sezione di ricerca.
  La visualizzazione dei profili cambia per tipologia di iscritto:
  \begin{itemize}

  \item Basic: visualizzazione delle informazioni generali (nickName,
    nome, cognome, data di nascita) e delle amicizie
  \item Business: stessa vista dei membri Basic con l'aggiunta
    degli Hobby e degli Interessi
  \item Executive: stessa vista dei membri Business con l'aggiunta
    delle Esperienze lavorative

  \end{itemize}

\item Gestione delle proprie amicizie con possibilit\`a di aggiunta e
  rimozione
  
\end{itemize}

L'amministratore di LinQedIn ha a disposizione le seguenti funzionalit\`a:
\begin{itemize}

\item Creazione di un nuovo membro con la possibilit\`a di scelta di tipo
  del nuovo utente e con una procedura guidata per l'aggiunta delle
  informazioni principali.

\item Ricerca degli iscritti al servizio con possibilit\`a di visualizzare
  il loro profilo

\item Rimozione di un membro iscritto

\item Cambiamento di tipologia di un utente Iscritto

\item Salvataggio del Database

\end{itemize}

LinQedIn si appoggia su un database scritto nel formato xml, dove vengono
salvate le informazioni relative agli Iscritti e le loro amicizie.

